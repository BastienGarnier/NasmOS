\documentclass{book}
\usepackage[top=3cm, bottom=3cm, left=3cm, right=3cm]{geometry}
\usepackage{graphicx}
\usepackage{cancel}
\usepackage{tabto}
\usepackage{xparse}
\usepackage{amssymb}
\usepackage{amsmath}
\usepackage{hyperref}
\usepackage[newfloat]{minted}
\usepackage{caption}
\usepackage{listings}
\graphicspath{{./img/}}

\newcommand*{\addheight}[2][.5ex]{%
  \raisebox{0pt}[\dimexpr\height+(#1)\relax]{#2}%
}

\newcounter{compteur_definitions}
\setcounter{compteur_definitions}{0}

\DeclareDocumentCommand\definition{g}{
	\stepcounter{compteur_definitions}
	\textbf{Définition \arabic{compteur_definitions}\IfNoValueF{#1}{ (#1)}.}
}

\newcounter{compteur_propriete}
\setcounter{compteur_propriete}{0}

\DeclareDocumentCommand\property{g}{
	\stepcounter{compteur_propriete}
	\textbf{Propriété \arabic{compteur_propriete}\IfNoValueF{#1}{ (#1)}.}
}

\newcommand*{\jumpline}{\newline \newline}

\setlength{\parindent}{0pt}

\begin{document}
\title{Documentation OS}
\maketitle
\chapter{Introduction}
On suppose pour toute la suite que le système utilise un processeur Intel 32 bits.
\section{Émulation}
Pour compiler :
\begin{minted}[frame=single]{bash}
./building.sh
\end{minted}
Pour émuler en disquette :
\begin{minted}[frame=single]{bash}
qemu-system-x86_64 -boot a -drive file=floppyA,format=raw,index=0,media=disk
\end{minted}
\section{Normes de programmation}
Les normes décrites ci-dessous ont comme principalement but d'assurer la stabilité des programmes écrits, la lisibilité des codes et ainsi la facilitation du déboguage.
\jumpline 
L'unique langage de programmation utilisé est l'assembleur NASM (\url{https://www.nasm.us/}).
\subsection{Conventions pour la modularité}
Tout module est composé de deux fichiers :
\begin{itemize}
	\item un fichier de code d'extension \textit{.ASM}
	\item un fichier d'entête d'extension \textit{.INC}
\end{itemize}
Le fichier d'entête est systématiquement inclu par directive de préprocesseur au début du fichier de code.
Après l'écriture des directives de préprocesseur et avant le code lui-même, le fichier de code doit avoir une ligne \textit{[BITS 32]} spécifiant à l'assembleur que l'assemblage doit être fait en 32 bits.
\jumpline
Le fichier d'entête doit se charger en totalité de l'import des fonctions externes nécessaires au fichier de code, de la définition des constantes, et de la globalisation des fonctions exportés par le module (et aucune autre).
\jumpline
\textbf{Remarque :} Toutes les fonctions externalisées doivent être nommés selon un espace de nom sous la forme : \textit{ESPACE\_\_FONCTION}.
\jumpline
\textbf{Exemple : }
\begin{lstlisting}[title=module.inc]
\end{lstlisting}
\begin{minted}[linenos,frame=single]{nasm}
%define CONST_INUTILE 0x1234

EXTERN screen__print
GLOBAL module__helloworld ; void -> void
\end{minted}

\begin{lstlisting}[title=module.asm]
\end{lstlisting}
\begin{minted}[linenos,frame=single]{nasm}
%include "module.inc"

[BITS 32]

msg db "Hello World !", 10, 0

; Fonction helloworld : void -> void
; Affiche le message "Hello World !" sur l'écran
module__helloworld:
	push ebp
	mov ebp, esp
		push msg
		call screen__print
	mov esp, ebp
	pop ebp
ret
\end{minted}
Dans le code ci-dessus, \textit{screen} et \textit{module} sont des espaces de nom.
\textbf{Import de fonctions externes :} Toutes les fonctions externes d'un même espace de nom sont importés sur la même ligne :
\begin{minted}[linenos,frame=single]{nasm}
EXTERN espace__f1, espace__f2, espace__f3 ; etc...
\end{minted}
\textbf{Globalisation :} Toutes les fonctions globalisées le sont sur des lignes séparées, toujours suivies par leur signature en commentaire :
\begin{minted}[linenos,frame=single]{nasm}
GLOBAL math__add ; int a, int b -> int
GLOBAL math__exp ; double x, unsigned int n -> double
GLOBAL math__abs ; int n -> unsigned int
\end{minted}
\textbf{Commentaire des fonctions :} Le fichier de code doit comprendre la description de chaque fonction en préambule de celle-ci :
\begin{minted}[linenos,frame=single]{nasm}
; Fonction nom_de_la_fonction_sans_espace_de_nom : TYPE a1, ..., TYPE an -> TYPE
; Description de l'action effectuée par cette fonction
; Parametres
; - a1 : sens de a1
; ...
; - an : sens de an
; Retour
; Sens du retour
espace__nom_de_la_fonction_sans_espace_de_nom:
	; CODE DE LA FONCTION
ret
\end{minted}
\subsection{Conventions d'appel de fonction}
Soient $n, k\in{\mathbb{N}}$ et $f$ une fonction \textit{n-aire} de paramètres $(p_{1}, \dots, p_{n})$.\newline
Les arguments lors de l'appel de cette fonction doivent être poussés sur la pile d'exécution du processus en cours dans l'ordre inverse des paramètres, de sorte que les adresses en pile correspondant à chaque paramètre conservent l'ordre de ceux-ci.
\jumpline
La valeur de renvoi de la fonction suit la norme du langage C :
\begin{itemize}
	\item Si la valeur de renvoi est stockée sur au plus 32 bits, elle est contenu dans $EAX$
	\item Si la valeur de renvoi est stockée sur au plus 64 bits, elle est contenu dans $(EDX:EAX)$
\end{itemize}
Exemple :
\begin{minted}[linenos,frame=single]{nasm}
section .text

_start:

	push dword 0 ; Argument 2
	push dword 0 ; Argument 1
	call fonction
	; EAX contient le résultat

end:
	jmp end
\end{minted}
\subsection{Conventions d'écriture de fonction}
Avant de créer une nouvelle trame d'exécution, on pousse $EAX$ en mémoire pour avec les 4 octets nécessaires à la valeur de renvoi. L'adresse en pile d'exécution de ce bloc de 4 octets contiendra la valeur de renvoi.
\jumpline
On crée ensuite une trame d'exécution :
\begin{minted}[linenos,frame=single]{nasm}
push ebp
mov ebp, esp
	; CODE DE LA FONCTION
mov esp, ebp
pop ebp
\end{minted}
Cependant, dans l'ordre actuel des choses il y a plusieurs problèmes :
\begin{itemize}
	\item Le bloc de renvoi n'a pas été réassigné à $EAX$
	\item Les blocs de renvoi et d'arguments en pile d'exécution n'ont pas été pop et surchargent donc la pile d'exécution
\end{itemize}
On règle ces deux problèmes d'un coup à la suite de la sortie de la trame :
\begin{minted}[linenos,frame=single]{nasm}
mov eax, [esp + 4]
mov [esp + N], eax ; N = 4 * n
pop eax
add esp, M ; M = 4 * (n - 1) = N - 4 ; Cette ligne peut être supprimée si M = 0
\end{minted}
\textbf{Remarque : } Ces problèmes n'existent que si $n > 1$. D'autre part, l'ajout des lignes juste ci-dessus implique un bug dans le cas où $n = 0$ (pas de paramètres). Pour cette raison, dans le cas où $1\geq{n}\geq{0}$, on écrira seulement :
\begin{minted}[linenos,frame=single]{nasm}
pop eax
\end{minted}
\textbf{Remarque :} Si la fonction ne renvoie rien, il n'est pas \textit{tout le temps} nécessaire de créer un espace pour la valeur de renvoi. Cependant, il est très régulier que les fonctions écrites utilisent le registre $EAX$. À ce moment là, la création d'un espace pour la valeur de renvoi est équivalent à la mémorisation de $EAX$.
\jumpline
En effet, il est souhaitable que la fonction écrite ne modifie pas les registres qu'elle utilise. Ceux-ci sont donc sauvegardés puis rechargés respectivement au début et à la fin de la trame nouvellement créée.
\jumpline
\textbf{Remarque : } Il n'est pas utile de mémoriser une nouvelle fois la valeur du registre $EAX$, voir remarque précédente.
\jumpline
\textbf{Exemples : }
\begin{minted}[linenos,frame=single]{nasm}
section .text

_start:

	push dword 20
	push dword 10
	call fonction_2_parametres_ou_plus

	push dword 42
	call fonction_1_parametre_ou_moins

end:
	jmp end

fonction_2_parametres_ou_plus:
	push eax ; crée un espace sur la pile pour la valeur de renvoi
	push ebp
	mov ebp, esp
	push ebx ; sauvegarde du registre EBX
		mov eax, [ebp + 12] ; Le premier argument vaut 10
		mov ebx, [ebp + 16] ; Le second argument vaut 20
		add eax, ebx
		mov [ebp + 4], eax ; Valeur de retour modifiée
	pop ebx ; chargement du registre EBX
	mov esp, ebp
	pop ebp
	mov eax, [esp + 4]
	mov [esp + 8], eax
	pop eax
	add esp, 4
ret

fonction_1_parametre_ou_moins:
	push eax
	push ebp
	mov ebp, esp
		mov eax, [ebp + 12]
		inc eax
		mov [ebp + 4], eax
	mov esp, ebp
	pop ebp
	pop eax
ret
\end{minted}
Ainsi, comme les registres sont de 32 bits soit 4 octets, les arguments sont aux adresses suivantes :
\begin{itemize}
	\item $[ebp + 12]$, pour $p_{1}$
	\item $[ebp + 16]$, pour $p_{2}$
	\item \dots
	\item $[ebp + 8 + 4 * i]$, pour $p_{i}$ le $i^{e}$ argument donné à la fonction
\end{itemize}
\section{BootSector}
\chapter{Kernel}
\section{Global Descriptor Table}
\subsection{Descripteur de segment}
\subsection{Table des descripteurs}
\section{Interruptions Descriptor Table}
\section{Microcontrôleurs PIC}
\section{Pagination}
\subsection{Buddy Memory Allocation}
\textbf{Minimal :}
\jumpline
Nombre minimal de bits nécessaire pour le tableau de mapping : (où $p$ désigne le nombre de niveaux)
$$N_{bits}(p) = \lceil{log_{2}(p + 2)}\rceil + \displaystyle\sum_{i = 0}^{p}\left(2^{p-i}\lceil{log_{2}(i+1)}\rceil\right)$$
En particulier comme on manipule $2^{20}$ blocs, il faut $N_{bits}(20) = 1712160$, soit $N_{octets}(20) = \dfrac{N_{bits}(20)}{8} = 214020$
\jumpline
En effet, il doit être précisé pour chaque case quel est le niveau de réservation. La première case peut réserver jusqu'à 20 niveaux. La case du milieu jusqu'à 19 niveaux, etc\dots Il faut donc stocker chacune des cases sur un nombre de bits différents.
\jumpline
\textbf{Utilisé :}
\jumpline
On note l'ordre d'un noeud $nd$ d'un arbre $t$ : $$o(nd) = h(t) - h(nd)\text{, où }h\text{ est la fonction hauteur}$$
\property Si l'arbre est complet, $2^{o(nd)}$ est le nombre de feuilles découlant de ce noeud.
\jumpline
On construit un arbre pour les $2^{20}$ blocs. On a donc $2^{21}$ noeuds. Chaque noeud contient le plus grand entier $p$ tel que les pages d'un sous-noeud d'ordre $p$ soient libres.
\jumpline
On a donc besoin de (où $2^{p}$ est le nombre de pages) :
$$N_{bits}(2^{p}) = 2^{p} + \displaystyle\sum_{i = 1}^{p}2^{p - i}\lceil{log_{2}(i + 1)}\rceil$$
En particulier, pour $2^{20}$ pages, on a besoin de 2760731 bits, soit 345092 octets.
\section{Appels systèmes}
\end{document}